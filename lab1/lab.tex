\def\VCDate{2017/09/08}\def\VCVersion{(Current)}
\documentclass{ffslides}
\ffpage{25}{\numexpr 16/9}
\usepackage{ProofPower}
\begin{document}
\obeyspaces

\normalpage{cs113 Lab1::By: Amuldeep Dhillon}{
cs113 lab1
\begin{GFT}{Text written to file build.sh}
\+doctex lab.doc\\
\+pptexenv latex lab.tex\\
\+dvipdf lab.dvi\\
\end{GFT}
\begin{GFT}{Bourne Shell}
\+chmod 777 build.sh\\
\end{GFT}
}


\normalpage{ch1 Proposiational Calculus (Conjuntion)}{
Exploring compound propositions

\qi{Conjunction: $p \land q$}

\begin{tabular}{|c|c||c|}
\hline
$ p $ & $ q $ & $ (p \wedge q) $ \\
\hline
F & F & F \\
F & T & F \\
T & F & F \\
T & T & T \\
\hline
\end{tabular}
\label{table:tt1}

\begin{GFT}{SML}
\+fun AND (false,x) = false | AND (true,x) = x;\\
\+val truthValues = [(false,false),(false,true),(true,false),(true,true)];\\
\+map AND truthValues;\\
\end{GFT}
}
\putfig{.4}{.2}{.5}{fig1}




\normalpage{ch1 Proposiational Calculus (Disjunction)}{
Exploring compound propositions

\qi{Disjunction: $p \lor q$}

\begin{tabular}{|c|c||c|}
\hline
$ p $ & $ q $ & $ (p \wedge q) $ \\
\hline
F & F & F \\
F & T & T \\
T & F & T \\
T & T & T \\
\hline
\end{tabular}
\label{table:tt1}

\begin{GFT}{SML}
\+fun OR (true,x) = true | OR (false,x) = x;\\
\+val truthValues = [(false,false),(false,true),(true,false),(true,true)];\\
\+map OR truthValues;\\
\end{GFT}
}
\putfig{.4}{.2}{.5}{fig2}


\normalpage{ch1 Proposiational Calculus (Negation)}{
Exploring compound propositions

\qi{Negation: $\sim p$}

\begin{tabular}{|c||c|}
\hline
$ p $ & $  \sim p $ \\
\hline
F & T \\
T & F \\
\hline
\end{tabular}
\label{table:tt1}


\begin{GFT}{SML}
\+fun NOT (true) = false | NOT(false) = true;\\
\+val truthValues = [(false),(true)];\\
\+map NOT truthValues;\\
\end{GFT}
}
\putfig{.41}{.2}{.5}{fig3}


\normalpage{ch1 Proposiational Calculus (Exclusive OR)}{
Exploring compound propositions

\qi{Exclusive OR: $p \oplus q$}

\begin{tabular}{|c|c||c|}
\hline
$ p $ & $ q $ & $ (p \oplus q) $ \\
\hline
F & F & F \\
F & T & T \\
T & F & T \\
T & T & F \\
\hline
\end{tabular}
\label{table:tt1}

\begin{GFT}{SML}
\+fun XOR(false,x) = x | XOR (true,x) = NOT x;\\
\+val truthValues = [(false,false),(false,true),(true,false),(true,true)];\\
\+map XOR truthValues;\\
\end{GFT}
}
\putfig{.4}{.2}{.5}{fig4}


\normalpage{ch1 Proposiational Calculus (Problem 1.18)}{
Using compound propositions

\qi{$p \oplus q \equiv (p \lor q) \land \sim (p \land q)$}

\begin{tabular}{|c|c||c||c|}
\hline
$ p $ & $ q $ & $ (p \oplus q) $ & $ ((p \vee q) \wedge  \sim (p \wedge q)) $\\
\hline
F & F & F & F \\
F & T & T & T \\
T & F & T & T \\
T & T & F & F \\
\hline
\end{tabular}
\label{table:tt1}

\qii{Since the last two colums of the truth table are the same, the two propositions are equivalent}

\begin{GFT}{SML}
\+fun f1(x,y) = XOR(x,y);\\
\+fun f2(x,y) = AND(OR(x,y),NOT(AND(x,y)));\\
\+map f1 truthValues;\\
\+map f2 truthValues;\\
\end{GFT}
}
\putfig{.4}{.11}{.5}{fig5}


\normalpage{ch1 Proposiational Calculus (Problem 1.13 a Part 1)}{
Using compound propositions

\qi{Simplify $p \oplus p$ and $p \oplus (p \oplus p)$}

\begin{tabular}{|c||c||c|}
\hline
$ p $ & $ (p \oplus p) $ & $ F $\\
\hline
F & F & F \\
T & F & F \\
\hline
\end{tabular}
\label{table:tt1}

\qii{Since the last two columns, "F" and "$ p \oplus p $" are identical the propositions are equivalent}

\begin{GFT}{SML}
\+val truthValues2 = [(false,false),(true,true)];\\
\+fun p13a1(x,y) = XOR(x,y);\\
\+fun FALSE(x,y) = false;\\
\+map p13a11 truthValues2;\\
\+map p13a12 truthValues2;\\
\end{GFT}
}
\putfig{.4}{.5}{.5}{fig6}



\normalpage{ch1 Proposiational Calculus (Problem 1.13 a Part 2)}{
Using compound propositions

\qi{Simplify $p \oplus p$ and $p \oplus (p \oplus p)$}

\begin{tabular}{|c||c|}
\hline
$ p $ & $ p \oplus (p \oplus p) $\\
\hline
F & F \\
T & T \\
\hline
\end{tabular}
\label{table:tt1}

\qii{Since the column "p" and "$p \oplus (p \oplus p) $ are identical the propositions are equivalent}

\begin{GFT}{SML}
\+val truthValues2 = [(false,false),(true,true)];\\
\+fun p13a21(x,y) = XOR(x,XOR(x,y));\\
\+fun p13a22(x,y) = x;\\
\+map p13a21 truthValues2;\\
\+map p13a22 truthValues2;\\
\end{GFT}
}
\putfig{.4}{.5}{.5}{fig62}



\normalpage{ch1 Proposiational Calculus (Problem 1.13 b)}{
Using compound propositions

\qi{Is $(p \oplus q) \oplus r \equiv p \oplus (q \oplus r)$}

\begin{tabular}{|c|c|c||c||c|}
\hline
$ p $ & $ q $ & $ r $ & $ ((p \oplus q) \oplus r) $  & $ (p \oplus (q \oplus r)) $ \\
\hline
F & F & F & F & F \\
F & F & T & T & T \\
F & T & F & T & T \\
F & T & T & F & F \\
T & F & F & T & T \\
T & F & T & F & F \\
T & T & F & F & F \\
T & T & T & T & T\\
\hline
\end{tabular}
\label{table:tt1}


\qii{Since the last columns are the same \\then the respective propositions are \\equivalent}

}

\ctext{.4}{.5}{.4}{
\begin{GFT}{SML}
\+fun f13b1(x,y,z) = XOR(z,XOR(x,y));\\
\+fun f13b2(x,y,z) = XOR(x,XOR(y,z));\\
\+val truthValues3 = [(false,false,false),(false,false,true),\\
\+(false,true,false),(false,true,true),(true,false,false),\\
\+(true,false,true),(true,true,false),(true,true,true)];\\
\+map f13b1 truthValues3;\\
\+map f13b2 truthValues3;\\
\end{GFT}
}
\putfig{.4}{.1}{.5}{fig7}




\normalpage{ch1 Proposiational Calculus (Problem 1.13 c)}{
Using compound propositions

\qi{Is $(p \oplus q) \land r \equiv p(p \land r) \oplus (q \land r)$}

\begin{tabular}{|c|c|c||c||c|}
\hline
$ p $ & $ q $ & $ r $ & $ ((p \oplus q) \wedge r) $ & $ ((p \wedge r) \oplus (q \wedge r)) $\\
\hline
F & F & F & F & F \\
F & F & T & F & F \\
F & T & F & F & F \\
F & T & T & T & T \\
T & F & F & F & F \\
T & F & T & T & T \\
T & T & F & F & F \\
T & T & T & F & F \\
\hline
\end{tabular}
\label{table:tt1}



\qii{Since the last columns are the same \\then the respective propositions are \\ equivalent}

}

\ctext{.45}{.5}{.4}{
\begin{GFT}{SML}
\+fun f13c1(x,y,z) = AND(XOR(x,y),z);\\
\+fun f13c2(x,y,z) = XOR(AND(x,z),AND(y,z));\\
\+map f13b1 truthValues3;\\
\+map f13b2 truthValues3;\\
\end{GFT}
}
\putfig{.45}{.1}{.5}{fig8}





\normalpage{ch1 Proposiational Calculus (Problem 1.17)}{
Using compound propositions

\qi{Construct the truth table for $(p \land q) \lor (\sim p)$}

\label{table:tt1}\begin{tabular}{|c|c||c|c|c|}
\hline
$ p $ & $ q $ & $ (p \wedge q) $ & $  \sim p $ & $ ((p \wedge q) \vee  \sim p) $ \\
\hline
F & F & F & T & T \\
F & T & F & T & T \\
T & F & F & F & F \\
T & T & T & F & T \\
\hline
\end{tabular}
\label{table:tt1}



\begin{GFT}{SML}
\+fun p17(x,y) = OR(AND(x,y),NOT(x));\\
\+map p17 truthValues;\\
\end{GFT}
}
\putfig{.45}{.2}{.5}{fig9}


\normalpage{ch1 Proposiational Calculus (Problem 1.19)}{
Using compound propositions

\qi{Show that $p \lor \sim(p \land q)$ is a tautology}

\begin{tabular}{|c|c||c|}
\hline
$ p $ & $ q $ & $ (p \vee  \sim (p \wedge q)) $ \\
\hline
F & F & T \\
F & T & T \\
T & F & T \\
T & T & T \\
\hline
\end{tabular}
\label{table:tt1}


\qii{Since the last column of the truth table is always "T", then the proposition is a tautology}


\begin{GFT}{SML}
\+fun p19(x,y) = OR(x,NOT(AND(x,y)));\\
\+map p19 truthValues;\\
\end{GFT}
}
\putfig{.4}{.2}{.5}{fig10}




\normalpage{ch1 Proposiational Calculus (Problem 1.20)}{
Using compound propositions

\qi{Show that $\sim p \land (p \land q)$ is a contradiction}

\begin{tabular}{|c|c||c|}
\hline
$ p $ & $ q $ & $ ( \sim p \wedge (p \wedge q)) $ \\
\hline
F & F & F \\
F & T & F \\
T & F & F \\
T & T & F \\
\hline
\end{tabular}
\label{table:tt1}


\qii{Since the last column of the truth table is always "F", then the proposition is a contradiction}


\begin{GFT}{SML}
\+fun p20(x,y) = AND(NOT(x),AND(x,y));\\
\+map p20 truthValues;\\
\end{GFT}
}
\putfig{.4}{.2}{.5}{fig11}


\end{document}


