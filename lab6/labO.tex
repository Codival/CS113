\def\VCDate{2017/11/17}\def\VCVersion{(Current)}
\documentclass{ffslides}
\ffpage{25}{\numexpr 16/9}
\usepackage{ProofPower,amsmath,mathtools,amssymb,venndiagram,verbatim}
\begin{document}
\obeyspaces


\normalpage{cs113 Lab 5 By: Amuldeep Dhillon Build Script}{
\begin{GFT}{Text written to file build.sh}
\+doctex labO.doc\\
\+pptexenv latex labO.tex\\
\+dvipdf labO.dvi\\
\end{GFT}
\begin{GFT}{Bourne Shell}
\+chmod 777 build.sh\\
\end{GFT}
}




\begin{comment}
\normalpage{All the SML}{



\begin{GFT}{SML}
\+(* how to define a set in SML? *)\\
\+(* use a list *)\\
\+val A = [3,1,2]; val B = []; val C = [2,3,1]; val D = [2,3,3,1,4];\\
\+fun empty([]) = true | empty(\_) = false;\\
\+empty(A); empty(B);\\
\+(* determine if some value is an element of a set *)\\
\+fun element(e,[]) = false \\
\+  | element(e,(x::xs)) = x=e orelse element(e,(xs));\\
\+element(3,A); element(4,A); element(4,C); element(3,C);\\
\+\\
\+(*Is one set a subset of another*)\\
\+fun subset([],[]) = true\\
\+  | subset([],(x::xs)) = true\\
\+  | subset((x::xs),[]) = false\\
\+  | subset((x::xs),(y::ys)) = element(x,(y::ys)) andalso subset(xs,(y::ys));\\
\+subset(A,C); subset(B,A); subset(A,B); subset(A,D); subset(C,D);\\
\+\\
\+\\
\+\\
\+(* Are two sets equal *)\\
\+fun equal(xs,ys) = subset(xs,ys) andalso subset(ys,xs);\\
\+equal(A,B); equal(A,C); equal(A,D);\\
\+val A = [1,3,5]; val B = [5,3,1];\\
\+equal(A,B);\\
\+fun properSubset(x,y) = subset(x,y) andalso not (equal(x,y));\\
\+subset(A,B);\\
\+properSubset(A,B);\\
\+val x = 5;\\
\+val A = []; val B = [x];\\
\+subset(A,B);\\
\+\\
\+(* powerset set of all subsets *)\\
\+PolyML.print\_depth 128;\\
\+val A = [1,3,5]; val B = [1,2]; \\
\+fun union([],ys) = ys | union(x::xs,ys) = \\
\+                        if element(x,ys) then union(xs,ys)\\
\+                        else x::union(xs,ys);\\
\+union(A,B);\\
\+fun insert(x,[]) = [] | insert(x,y::ys) = union([x],y)::insert(x,ys);\\
\+insert(3,[B]);\\
\+fun powerset([]) = [[]] | powerset(x::xs) =\\
\+                          union(insert(x,powerset(xs)),powerset(xs));\\
\+powerset(A);\\
\+(* cartesian product  set of pairs or tupels of any size *)\\
\+val A = [1,2]; val B = [1,2];\\
\+\\
\+fun onepair(x,[]) = [] | onepair(x,y::ys) = (x,y)::onepair(x,ys)\\
\+\\
\+fun cartesian([],ys) = []\\
\+   |cartesian(x::xs,ys) = onepair(x,ys) @ cartesian(xs,ys);\\
\+\\
\+cartesian(A,B);\\
\+cartesian(A,C);\\
\+\\
\+\\
\+\\
\+\\
\end{GFT}
}
\end{comment}



\normalpage{Problem 13.6}{
\begin{GFT}{SML}
\+fun element(e,[]) = false \\
\+  | element(e,(x::xs)) = x=e orelse element(e,(xs));\\
\+fun subset([],[]) = true\\
\+  | subset([],(x::xs)) = true\\
\+  | subset((x::xs),[]) = false\\
\+  | subset((x::xs),(y::ys)) = element(x,(y::ys)) andalso subset(xs,(y::ys));\\
\+fun union([],ys) = ys | union(x::xs,ys) = \\
\+                        if element(x,ys) then union(xs,ys)\\
\+                        else x::union(xs,ys);\\
\+fun onepair(x,[]) = [] | onepair(x,y::ys) = (x,y)::onepair(x,ys);\\
\+fun cartesian([],ys) = []\\
\+   |cartesian(x::xs,ys) = onepair(x,ys) @ cartesian(xs,ys);\\
\+fun equal(xs,ys) = subset(xs,ys) andalso subset(ys,xs);\\
\end{GFT}
}
\putfig{.5}{.2}{.4}{prob13-6}
\normalpage{Problem 13.6 cont.}{
\begin{GFT}{SML}
\+val A = [1,2,3]; val B = [3,6,9]; val C = [5,10,15];\\
\+equal(cartesian(A,union(B,C)),union(cartesian(A,B),cartesian(A,C)));\\
\+val A = [3,6,4,5]; val B = [3,4]; val C = [3,3,3];\\
\+equal(cartesian(A,union(B,C)),union(cartesian(A,B),cartesian(A,C)));\\
\+val A = [45,65,78,5,4,2,11]; val B = [23,1234,5543,57]; val C = [1];\\
\+equal(cartesian(A,union(B,C)),union(cartesian(A,B),cartesian(A,C)));\\
\end{GFT}
\qi{$\therefore$ A x (B $\cup$ C) = (A x B) $\cup$ (A x C)}
}
\putfig{.5}{.5}{.4}{prob13-6}
\normalpage{Problem 13.6 Test Cases}{
}

\putfig{.1}{.2}{.35}{13-6SML1}
\putfig{.5}{.2}{.35}{13-6SML2}
\putfig{.5}{.6}{.35}{13-6SML3}



\normalpage{Problem 12.10}{
\begin{GFT}{SML}
\+fun count([]) = 0 | count((\_,c)::xs) = c + count(xs);\\
\+fun subtractList(ys,[]) = count(ys) | subtractList(xs,ys) =\\
\+                           count(xs) - count(ys);\\
\+fun addList(ys,[]) = count(ys) | addList(xs,ys) = \\
\+                      count(xs) + count (ys);\\
\+fun subtract(ys,x) = count(ys) - x;\\
\+fun add(ys,x) = count(ys) + x;\\
\+\\
\+val total = [("All",60)];\\
\+val tomato = [("tomato",45)];\\
\+val both = [("tomato\&onion",30)];\\
\+val plain = [("plain",5)];\\
\+val notOnions = addList(tomato,plain);\\
\+val OnionOrTomato = add(tomato,subtract(total,notOnions));\\
\+val justOnions = subtract(total,notOnions);\\
\+val Onions = add(both,justOnions);\\
\end{GFT}
}
\putfig{.5}{.3}{.4}{prob12-10}

\normalpage{Problem 12.10 cont.}{
\begin{GFT}{SML}
\+val total = [("All",20)];\\
\+val tomato = [("tomato",5)];\\
\+val both = [("tomato\&onion",5)];\\
\+val plain = [("plain",0)];\\
\+val notOnions = addList(tomato,plain);\\
\+val OnionOrTomato = add(tomato,subtract(total,notOnions));\\
\+val justOnions = subtract(total,notOnions);\\
\+val Onions = add(both,justOnions);\\
\+val total = [("All",100)];\\
\+val tomato = [("tomato",50)];\\
\+val both = [("tomato\&onion",40)];\\
\+val plain = [("plain",10)];\\
\+val notOnions = addList(tomato,plain);\\
\+val OnionOrTomato = add(tomato,subtract(total,notOnions));\\
\+val justOnions = subtract(total,notOnions);\\
\+val Onions = add(both,justOnions);\\
\end{GFT}
}
\putfig{.5}{.5}{.4}{prob12-10}

\normalpage{Problem 12.10 Test Cases}{
\qi{If you have 60 sandwhiches, 45 with tomato, 30 with tomato and onion, and 5 with neither}
\qii{Then there are 55 with tomatoes or onions, 40 with onions, and 10 with only onion}
\qi{If you have 20 sandwhiches, 5 with tomato, 5 with tomato and onion, and 0 with neither}
\qii{Then there are 20 with tomatoes or onions, 20 with onions, and 15 with only onion}
\qi{If you have 100 sandwhiches, 50 with tomato, 40 with tomato and onion, and 10 with neither}
\qii{Then there are 90 with tomatoes or onions, 80 with onions, and 40 with only onion}
}
\putfig{.05}{.4}{.4}{12-10SML1}
\putfig{.5}{.4}{.4}{12-10SML2}

\normalpage{Problem 12.10 Visual}{
}
\putfig{.1}{.1}{.8}{venn}

\normalpage{Example 13.3}{
\begin{GFT}{SML}
\+fun inter([],ys) = [] | inter(x::xs,ys) =\\
\+                        if element(x,ys) then x::inter(xs,ys)\\
\+                        else inter(xs,ys);\\
\+val A = [1,2,3,4,5,6]; \\
\+val A1 = [1,2]; val A2 = [3,4]; val A3 = [5,6];\\
\+equal(union(union(A1,A2),A3),A);\\
\+inter(A1,A2);\\
\+inter(A2,A3);\\
\+inter(A1,A3);\\
\+val A = [1,5,6,7,8];\\
\+val A1 = [1,5,6]; val A2 = [7]; val A3 = [8];\\
\+equal(union(union(A1,A2),A3),A);\\
\+inter(A1,A2);\\
\+inter(A2,A3);\\
\+inter(A1,A3);\\
\end{GFT}
}
\putfig{.5}{.2}{.4}{prob13-3}
\normalpage{Example 13.3 cont.}{
\begin{GFT}{SML}
\+val A = [1,4,5,6,7,8,12];\\
\+val A1 =[1,4]; val A2 = [5]; val A3 = [6,7]; val A4 = [8]; val A5 = [12];\\
\+equal(union(union(union(union(A1,A2),A3),A4),A5),A);\\
\+inter(A1,A2);\\
\+inter(A1,A3);inter(A1,A4);inter(A1,A5);\\
\+inter(A2,A3);inter(A2,A4);inter(A2,A5);\\
\+inter(A3,A4);inter(A3,A5);inter(A4,A5);\\
\end{GFT}
\qi{A1 $\cup$ A2 $\cup$ A3 ... = A}
\qi{A1 $\cap$ A2 = A1 $\cap$ A3 = A2 $\cap$ A3 ... =$\emptyset$}
\qi{$\therefore$ by definition A1,A2,A3 ... are a partition of A }
}
\putfig{.5}{.4}{.4}{prob13-3}

\normalpage{Example 13.3 Test Cases}{
}

\putfig{.1}{.1}{.35}{13-3SML1}
\putfig{.1}{.5}{.35}{13-3SML2}
\putfig{.5}{.1}{.19}{13-3SML3}
\putfig{.5}{.4}{.19}{13-3SML4}



\end{document}
