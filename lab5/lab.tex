\def\VCDate{2017/11/09}\def\VCVersion{(Current)}
\documentclass{ffslides}
\ffpage{25}{\numexpr 16/9}
\usepackage{ProofPower,amsmath,mathtools,amssymb}
\usepackage{verbatim}
\begin{document}
\obeyspaces


\normalpage{cs113 Lab5: By Amuldeep Dhillon}{

\begin{GFT}{Text written to file build.sh}
\+doctex lab.doc\\
\+pptexenv latex lab.tex\\
\+dvipdf lab.dvi\\
\end{GFT}
\begin{GFT}{Bourne Shell}
\+chmod 777 build.sh\\
\+./build.sh\\
\end{GFT}
}

\begin{comment}
\normalpage{Ch 9 Introduction and Recursion}{
Implementation of some of the sequences and summations from the Stanford powerpoint slides.
\[\sum_{k=1}^n a_k  \equiv \frac{n(n+1)}{2}\]
\begin{GFT}{SML}
\+fun sum(k,n,f) = if k > n then 0 else f(k) + sum(k+1,n,f);\\
\+fun id(x) = x;\\
\+fun f(n) = n*(n+1) div 2;\\
\+val a = f(500);\\
\+val b = sum(1,500,id);\\
\+val result = a = b;\\
\end{GFT}
}
\putfig{.5}{.5}{.4}{ex}






\normalpage{Ch 9 Introduction and Recursion}{
Implementation of some of the sequences and summations from the Stanford powerpoint slides.

\begin{GFT}{SML}
\+val p = "Problem 1";\\
\+fun f1(k) = k*k + 2;\\
\+fun s1(1) = 3 | s1(k) = s1(k-1) + 2*k-1;\\
\+f1(5) = s1(5);\\
\end{GFT}
}



\normalpage{Ch 9 Introduction and Recursion}{
Implementation of some of the sequences and summations from the Stanford powerpoint slides.

\begin{GFT}{SML}
\+fun f1(k) = 7+4*(k-1);\\
\+fun f1(k) = 4*k +3;\\
\+fun s1(1) = 7 | s1(k) = s1(k-1) + 4;\\
\+f1(512) = s1(512);\\
\end{GFT}
}




\normalpage{Ch 9 Introduction and Recursion}{
Implementation of some of the sequences and summations from the Stanford powerpoint slides.
\[\sum_{k=1}^n a_k\]
\begin{GFT}{SML}
\+fun sum(k,n,f) = if k > n then 0 else f(k) + sum(k+1,n,f);\\
\+fun id(x) = x;\\
\+fun f(n) = n*(n+1) div 2;\\
\+f(500);\\
\+sum(1,500,id);\\
\end{GFT}
}




\normalpage{Ch 9 Introduction and Recursion}{
Implementation of some of the sequences and summations from the Stanford powerpoint slides.
\[\prod_{k=1}^n a_k\]
\begin{GFT}{SML}
\+fun prod(k,n,f) = if k > n then 1 else f(k) * prod(k+1,n,f);\\
\+fun id(x) = x;\\
\+fun f(n) = n*(n+1) div 2;\\
\+f(500);\\
\+prod(1,5,id);\\
\end{GFT}
}




\normalpage{Ch 9 Introduction and Recursion}{
Implementation of some of the sequences and summations from the Stanford powerpoint slides.
\[\sum_{k=1}^n  c  \cdot a_k + b_k\]
\begin{GFT}{SML}
\+val c =1; fun id(x) = x; fun zero(x) = 0;\\
\+fun sum(k,n,f1,f2) = if k > n then 0 else f(c,f1,f2,k) + sum(k+1,n,f1,f2);\\
\+fun term(c,f1,f2,k) = c * f1(k) + f2(k);\\
\+sum(1,10,id,term);\\
\+c * sum(1,5,id) + sum(1,5,zero);\\
\end{GFT}
}




\normalpage{Ch 9 Introduction and Recursion}{
Implementation of some of the sequences and summations from the Stanford powerpoint slides.
$ \sum_{i=1}^4  \sum_{j=1}^3 ij\ $ is equivalent to $ \sum_{i=1}^4  i \sum_{j=1}^3 j $
\begin{GFT}{SML}
\+fun f(i,j) = i*j;\\
\+fun sum(i,j,ni,nj,f) = \\
\+    let\\
\+        fun inner(j) = if j > nj then 0 else f(i,j) + inner(j+1);\\
\+    in\\
\+        if i > ni then 0 else inner(j) + sum(i+1,j,ni,nj,f)\\
\+    end;\\
\+sum(1,1,4,3,f);\\
\end{GFT}
}

\end{comment}




\normalpage{Ch 9 Problem 1}{
\begin{GFT}{SML}
\+fun f1(x) = (x*x) + x;\\
\+fun sum(k,n,f) = if k > n then 0 \\
\+    else (2*f(k)) + sum(k+1,n,f);\\
\+fun id(x) = x;\\
\+"Test Case P(5000)";\\
\+val a = sum(1,5000,id); val b = f1(5000); \\
\+val result = a = b;\\
\end{GFT}
}
\putfig{.05}{.55}{.5}{9-1}
\btext{.45}{.73}{.5}{\qi{Evaluate P(n) at P(n+1)}
\qi{2+4+...+2n = P(n)}
\qi{Induce induction hypothesis}
\qi{Associative property of addition \& Subtraction Property of Equality}
\qi{$n^2+2n+1 = (n+1)^2$}}
\putfig{.45}{.15}{.5}{p-1SML}







\normalpage{Ch 9 Problem 6}{
\begin{GFT}{SML}
\+fun f1(x) = ((x*x) + x) div 2 -3;\\
\+fun sum(k,n,f) = if k > n then \Twiddles{}3 \\
\+    else f(k) + sum(k+1,n,f);\\
\+fun id(x) = x;\\
\+val a = f1(1000); val b = sum(1,1000,id);\\
\+val result = a = b;\\
\end{GFT}
$\sum_{k=1}^n k\ \equiv \frac{n(n+1)}{2}$ \\
$(\sum_{k=1}^n k)-3\ \equiv \frac{n(n+1)}{2}-3$\\
so, 3+4+5+6+...+1000 equals to the summation of integers \\
from 1 - 1000 minus 3
so we can use the equation $\frac{n(n+1)}{2}-3$\\
so the answer to the summation is 500497
}

\putfig{.45}{.07}{.4}{p-6SML}
\btext{.55}{.6}{.28}{
\qi{Given}
\qi{Subtraction Property of Equality}
\qi{Subtraction Property of Equality}
}









\normalpage{Ch 9 Problem 17}{
\begin{GFT}{SML}
\+fun f1(x) = (x*x);\\
\+fun sum(k,n,f) = if k > n then 0 \\
\+    else (2*f(k)-1) + sum(k+1,n,f);\\
\+fun id(x) = x;\\
\+"Test Case P(1500)";\\
\+val a = sum(1,1500,id); val b = f1(1500); \\
\+val result = a = b;\\
\end{GFT}
}
\btext{.5}{.6}{.3}{
\qi{evalute the funtion at n+1}
\qi{associative property of addition}
\qi{using the induction hypothesis}
\qi{factoring}}
\putfig{.05}{.55}{.39}{9-17SML}
\putfig{.45}{.15}{.5}{9-17}












\end{document}


