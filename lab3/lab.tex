\def\VCDate{2017/10/03}\def\VCVersion{(Current)}
\documentclass{ffslides}
\ffpage{25}{\numexpr 16/9}
\usepackage{ProofPower}
\usepackage{verbatim}
\usepackage{amsmath}
\begin{document}
\obeyspaces


\normalpage{cs113 Lab3: By Amuldeep Dhillon}{

\begin{GFT}{Text written to file build.sh}
\+doctex lab.doc\\
\+pptexenv latex lab.tex\\
\+dvipdf lab.dvi\\
\end{GFT}
\begin{GFT}{Bourne Shell}
\+chmod 777 build.sh\\
\+./build.sh\\
\end{GFT}
}

\normalpage{cs113 Lab3: Conditional}{

\begin{tabular}{|c|c||c|}
\hline
$ p $ & $ q $ & $ (p \rightarrow q) $ \\
\hline
F & F & T \\
F & T & T \\
T & F & F \\
T & T & T \\
\hline
\end{tabular}
\label{table:tt1}


\begin{GFT}{SML}
\+val truthValues2 = [(false,false),(false,true),(true,false),(true,true)];\\
\+fun COND(false,q) = true | COND(true,q) = q;\\
\+map COND truthValues2;\\
\+val f1 = map COND truthValues2;\\
\+val l1 = [(true),(true),(true)];\\
\+fun tautology [] = true | tautology (x::xs) = x andalso tautology xs;\\
\+tautology f1;\\
\+tautology l1;\\
\end{GFT}
}
%\putfig{.6}{.2}{.3}{notGate}
\putfig{.05}{.6}{.5}{conditionaltestSML}
%\btext{.68}{.5}{.12}{\LARGE $R \equiv \sim P$}




\normalpage{cs113 Lab3: Conditional}{
\qi{Use ttgen to show equivalence of \\
$p \rightarrow q$ and  $\lnot p \lor q$}

\begin{tabular}{|c|c||c|}
\hline
$ p $ & $ q $ & $ ( \sim p \vee q) $ \\
\hline
F & F & T \\
F & T & T \\
T & F & F \\
T & T & T \\
\hline
\end{tabular}
\label{table:tt1}
\begin{tabular}{|c|c||c|}
\hline
$ p $ & $ q $ & $ ((p \rightarrow q) \leftrightarrow ( \sim p \vee q)) $ \\
\hline
F & F & T \\
F & T & T \\
T & F & T \\
T & T & T \\
\hline
\end{tabular}
\label{table:tt1}


\begin{GFT}{SML}
\+val truthValues2 = [(false,false),(false,true),(true,false),(true,true)];\\
\+fun COND(false,q) = true | COND(true,q) = q;\\
\+map COND truthValues2;\\
\end{GFT}
}
%\putfig{.6}{.2}{.3}{notGate}
\putfig{.05}{.6}{.5}{conditionaltestSML}
%\btext{.68}{.5}{.12}{\LARGE $R \equiv \sim P$}







\normalpage{cs113 Lab3: DeMorgan's Theorem for Conditionals}{

\qi{$\sim(p \rightarrow q) \equiv p \land \sim q$}
\begin{equation}
\begin{split}
\sim(p \rightarrow q) & \equiv \sim((\sim p) \lor q)\\
                      & \equiv (\sim (\sim p)) \land \sim q\\
                      & \equiv p \land \sim q
\end{split}
\end{equation}
}







\normalpage{cs113 Lab3: Problem 3.11}{

\qi{$((p \lor q) \rightarrow r) \equiv (p \rightarrow r) \land (q \rightarrow r)$}
\[ Law1: A \rightarrow B \equiv \sim A \lor B \]
\begin{equation}
\begin{split}
((p \lor q) \rightarrow r) & \equiv (\sim (p \lor q)) \lor r \tag{Law1 with A + = (p \sim q)}\\
                            & \equiv (\sim (\sim p)) \land \sim q\\
                            & \equiv p \land \sim q
\end{split}
\end{equation}
}








\normalpage{cs113 Lab2: Biconditional}{

\begin{tabular}{|c|c||c|}
\hline
$ p $ & $ q $ & $ (p \rightarrow q) $ \\
\hline
F & F & T \\
F & T & T \\
T & F & F \\
T & T & T \\
\hline
\end{tabular}
\label{table:tt1}


\begin{GFT}{SML}
\+val truthValues2 = [(false,false),(false,true),(true,false),(true,true)];\\
\+fun BICOND(p,q) = p = q;\\
\+map BICOND truthValues2;\\
\+val f1 = map COND truthValues2;\\
\+val l1 = [(true),(true),(true)];\\
\+fun tautology [] = true | tautology (x::xs) = x andalso tautology xs;\\
\+tautology f1;\\
\+tautology l1;\\
\end{GFT}
}
%\putfig{.6}{.2}{.3}{notGate}
\putfig{.05}{.6}{.5}{conditionaltestSML}
%\btext{.68}{.5}{.12}{\LARGE $R \equiv \sim P$}

\normalpage{cs113 Lab3 Valid Arguements}{

\qi{Identify the premises and conclusion of the arguement}
\qi{Construct a truth table including premises and conclusion}
\qi{Find rows in which all premises are true}
\qi{In each row of previous step, if the conslusion is true then the
arguement is valid; otherwise it is invalid}

\begin{GFT}{SML}
\+fun p1(p,q) = COND(p,q);\\
\+fun p2(p,q) = q;\\
\+fun c(p,q) = p;\\
\+fun valid(p,q) = if p1(p,q) andalso p2(p,q) then c(p,q) else false;\\
\+tautology (map valid truthValues2);\\
\+\}\\
\+\\
\+\\
\+\\
\+\\
\+\Backslash{}normalpage\{cs113 Lab3 Valid Arguements\}\{\\
\+\\
\+\Backslash{}qi\{Identify the premises and conclusion of the arguement\}\\
\+\Backslash{}qi\{Construct a truth table including premises and conclusion\}\\
\+\Backslash{}qi\{Find rows in which all premises are true\}\\
\+\Backslash{}qi\{In each row of previous step, if the conslusion is true then the\\
\+arguement is valid; otherwise it is invalid\}\\
\+\\
\+Example 4.3\\
\+fun p1(p,q) = COND(p,q);\\
\+fun p2(p,q) = COND(q,p);\\
\+fun c(p,q) = p orelse q;\\
\+fun valid(p,q) = if p1(p,q) andalso p2(p,q) then c(p,q) else false;\\
\+tautology (map valid truthValues2);\\
\+\}\\
\+\\
\+\\
\+\\
\+\\
\+\Backslash{}normalpage\{cs113 Lab3 Valid Arguements\}\{\\
\+\\
\+\Backslash{}qi\{Identify the premises and conclusion of the arguement\}\\
\+\Backslash{}qi\{Construct a truth table including premises and conclusion\}\\
\+\Backslash{}qi\{Find rows in which all premises are true\}\\
\+\Backslash{}qi\{In each row of previous step, if the conslusion is true then the\\
\+arguement is valid; otherwise it is invalid\}\\
\+\\
\+Example 4.10\\
\+fun p1(p,q) = COND(p,q);\\
\+fun p2(p,q) = q;\\
\+fun c(p,q) = p;\\
\+fun valid(p,q) = if p1(p,q) andalso p2(p,q) then c(p,q) else false;\\
\+tautology (map valid truthValues2);\\
\end{GFT}
}



















































\begin{comment}



\normalpage{cs113 Lab2: And Gate}{

\begin{tabular}{|c|c||c|}
\hline
$ P $ & $ Q $ & $ P \wedge Q $ \\
\hline
0 & 0 & 0 \\
0 & 1 & 0 \\
1 & 0 & 0 \\
1 & 1 & 1 \\
\hline
\end{tabular}
\label{table:tt1}



\begin{GFT}{SML}
\+exception Error;\\
\+fun AND(false,\_) = false | AND(true,x) = x | AND(\_,\_) = raise Error;\\
\+val truthValues2 = [(false,false),(false,true),(true,false),(true,true)];\\
\+map AND truthValues2;\\
\end{GFT}
}
\putfig{.65}{.2}{.2}{andGate}
\putfig{.05}{.65}{.5}{andGateSML}
\btext{.67}{.6}{.15}{\LARGE $R \equiv P \land Q$}






\normalpage{cs113 Lab2: Or Gate}{

\begin{tabular}{|c|c||c|}
\hline
$ P $ & $ Q $ & $ P \vee Q $ \\
\hline
0 & 0 & 0 \\
0 & 1 & 1 \\
1 & 0 & 1 \\
1 & 1 & 1 \\
\hline
\end{tabular}
\label{table:tt1}



\begin{GFT}{SML}
\+exception Error;\\
\+fun OR(true,\_) = true | OR(false,x) = x | OR(\_,\_) = raise Error;\\
\+val truthValues2 = [(false,false),(false,true),(true,false),(true,true)];\\
\+map OR truthValues2;\\
\end{GFT}
}
\putfig{.65}{.2}{.2}{orGate}
\putfig{.05}{.65}{.5}{orGateSML}
\btext{.67}{.6}{.15}{\LARGE $R \equiv P \lor Q$}






\normalpage{cs113 Lab2: Nand Gate}{

\begin{tabular}{|c|c||c|}
\hline
$ P $ & $ Q $ & $  P \mid Q $ \\
\hline
0 & 0 & 1 \\
0 & 1 & 1 \\
1 & 0 & 1 \\
1 & 1 & 0 \\
\hline
\end{tabular}
\label{table:tt1}


\begin{GFT}{SML}
\+exception Error;\\
\+fun NAND(false,\_) = true | NAND(true,x) = NOT(x) | NAND(\_,\_) = raise Error;\\
\+val truthValues2 = [(false,false),(false,true),(true,false),(true,true)];\\
\+map NAND truthValues2;\\
\end{GFT}
}
\putfig{.65}{.2}{.2}{nandGate}
\putfig{.05}{.65}{.5}{nandGateSML}
\btext{.67}{.6}{.15}{\LARGE $R \equiv P \mid Q$}






\normalpage{cs113 Lab2: Nor Gate}{

\begin{tabular}{|c|c||c|}
\hline
$ P $ & $ Q $ & $   P \downarrow Q $ \\
\hline
0 & 0 & 1 \\
0 & 1 & 0 \\
1 & 0 & 0 \\
1 & 1 & 0 \\
\hline
\end{tabular}
\label{table:tt1}



\begin{GFT}{SML}
\+exception Error;\\
\+fun NOR(true,\_) = false | NOR(false,x) = NOT(x) | NOR(\_,\_) = raise Error;\\
\+val truthValues2 = [(false,false),(false,true),(true,false),(true,true)];\\
\+map NOR truthValues2;\\
\end{GFT}
}
\putfig{.65}{.2}{.2}{norGate}
\putfig{.05}{.65}{.5}{norGateSML}
\btext{.67}{.6}{.15}{\LARGE $R \equiv P \downarrow Q$}




\normalpage{cs113 Lab2: Xor Gate}{

\begin{tabular}{|c|c||c|}
\hline
$ P $ & $ Q $ & $ P \oplus Q $ \\
\hline
0 & 0 & 0 \\
0 & 1 & 1 \\
1 & 0 & 1 \\
1 & 1 & 0 \\
\hline
\end{tabular}
\label{table:tt1}



\begin{GFT}{SML}
\+exception Error;\\
\+fun XOR(false,x) = x | XOR(true,x) = NOT(x) | XOR(\_,\_) = raise Error;\\
\+val truthValues2 = [(false,false),(false,true),(true,false),(true,true)];\\
\+map XOR truthValues2;\\
\end{GFT}
}
\putfig{.65}{.2}{.25}{xorGate}
\putfig{.05}{.65}{.5}{xorGateSML}
\btext{.67}{.6}{.15}{\LARGE $R \equiv P \oplus Q$}






\normalpage{cs113 Lab2: Problem 2.1}{

\qi{Example 2.2 truth table and circuit}

\begin{tabular}{|c|c|c||c|}
\hline
$ P $ & $ Q $ & $ R $ & $ ((P \vee Q) \wedge (P \vee R)) $ \\
\hline
0 & 0 & 0 & 0 \\
0 & 0 & 1 & 0 \\
0 & 1 & 0 & 0 \\
0 & 1 & 1 & 1 \\
1 & 0 & 0 & 1 \\
1 & 0 & 1 & 1 \\
1 & 1 & 0 & 1 \\
1 & 1 & 1 & 1 \\
\hline
\end{tabular}
\label{table:tt1}



\begin{GFT}{SML}
\+fun ex22(P,Q,R) = AND(OR(P,Q),OR(P,R));\\
\+val truthValues3 = [(false,false,false),(false,false,true),(false,true,false),(false,true,true),\\
\+(true,false,false),(true,false,true),(true,true,false),(true,true,true)];\\
\+map ex22 truthValues3;\\
\end{GFT}
}
\putfig{.6}{.6}{.3}{ex22}
\putfig{.4}{.2}{.5}{ex22SML}
\btext{.5}{.5}{.3}{\LARGE $S \equiv ((P \vee Q) \wedge (P \vee R))$}






\normalpage{cs113 Lab2: Problem 2.2}{
\qi{Circuit of $((P \wedge Q) \vee  \sim R)$}
\begin{tabular}{|c|c|c||c|}
\hline
$ P $ & $ Q $ & $ R $ & $ ((P \wedge Q) \vee  \sim R) $ \\
\hline
0 & 0 & 0 & 1 \\
0 & 0 & 1 & 0 \\
0 & 1 & 0 & 1 \\
0 & 1 & 1 & 0 \\
1 & 0 & 0 & 1 \\
1 & 0 & 1 & 0 \\
1 & 1 & 0 & 1 \\
1 & 1 & 1 & 1 \\
\hline
\end{tabular}
\label{table:tt1}


\begin{GFT}{SML}
\+fun p22(P,Q,R) = OR(AND(P,Q),NOT(R));\\
\+val truthValues3 = [(0,0,0),(0,0,1),(0,1,0),(0,1,1),\\
\+(1,0,0),(1,0,1),(1,1,0),(1,1,1)];\\
\+map p22 truthValues3;\\
\end{GFT}
}
\putfig{.6}{.6}{.3}{p22}
\putfig{.4}{.2}{.5}{p22SML}
\btext{.5}{.5}{.3}{\LARGE $S \equiv ((P \wedge Q) \vee  \sim R)$}




\normalpage{cs113 Lab2: Problem 2.3}{
\qi{Match the figure in the Book for Problem 2.3}
\begin{tabular}{|c|c|c||c||c|}
\hline
$ P $ & $ Q $ & $ R $ & $ S $ & $ (P \land Q) \land (\sim R) $\\
\hline
1 & 1 & 1 & 0 & 0 \\
1 & 1 & 0 & 1 & 1 \\
1 & 0 & 1 & 0 & 0 \\
1 & 0 & 0 & 0 & 0 \\
0 & 1 & 1 & 0 & 0 \\
0 & 1 & 0 & 0 & 0 \\
0 & 0 & 1 & 0 & 0 \\
0 & 0 & 0 & 0 & 0\\
\hline
\end{tabular}
\label{table:tt1}

\qi{Since the last two columns are identical \\ (Left from textbook), they are equivalent}

\begin{GFT}{SML}
\+fun p23(P,Q,R) = AND(AND(P,Q),NOT(R));\\
\+val truthValues3 = [(0,0,0),(0,0,1),(0,1,0),(0,1,1),\\
\+(1,0,0),(1,0,1),(1,1,0),(1,1,1)];\\
\+map p23 truthValues3;\\
\end{GFT}
}
\putfig{.6}{.6}{.3}{p23}
\putfig{.4}{.2}{.5}{p23SML}
\btext{.5}{.5}{.3}{\LARGE $S \equiv ((P \land Q) \land  \sim R)$}






\normalpage{cs113 Lab2: Problem 2.7}{
\qi{Create the Truth Table for the given Circuit for Problem 2.7}



\begin{tabular}{|c|c|c||c|}
\hline
$ P $ & $ Q $ & $ R $ & $ (P \wedge  \sim Q) \vee (Q \wedge R) $ \\
\hline
0 & 0 & 0 & 0 \\
0 & 0 & 1 & 0 \\
0 & 1 & 0 & 0 \\
0 & 1 & 1 & 1 \\
1 & 0 & 0 & 1 \\
1 & 0 & 1 & 1 \\
1 & 1 & 0 & 0 \\
1 & 1 & 1 & 1 \\
\hline
\end{tabular}
\label{table:tt1}



\begin{GFT}{SML}
\+fun p27(P,Q,R) = OR(AND(P,NOT(Q)),AND(Q,R));\\
\+val truthValues3 = [(0,0,0),(0,0,1),(0,1,0),(0,1,1),\\
\+(1,0,0),(1,0,1),(1,1,0),(1,1,1)];\\
\+map p27 truthValues3;\\
\end{GFT}
}
\putfig{.6}{.6}{.3}{p27}
\putfig{.4}{.2}{.5}{p27SML}
\btext{.5}{.5}{.3}{\LARGE $S \equiv (P \wedge  \sim Q) \vee (Q \wedge R)$}









\normalpage{cs113 Lab2: Problem 2.8}{
\qi{Create the Truth Table for the exclusive nor for Problem 2.8}



\begin{tabular}{|c|c||c|}
\hline
$ P $ & $ Q $ & $  \sim (P \oplus Q) $ \\
\hline
0 & 0 & 1 \\
0 & 1 & 0 \\
1 & 0 & 0 \\
1 & 1 & 1 \\
\hline
\end{tabular}
\label{table:tt1}




\begin{GFT}{SML}
\+fun p28(P,Q) = NOT(XOR(P,Q));\\
\+val truthValues2 = [(0,0),(0,1),(1,0),(1,1)];\\
\+map p28 truthValues2;\\
\end{GFT}
}
\putfig{.5}{.6}{.3}{p28}
\putfig{.4}{.2}{.5}{p28SML}
\btext{.5}{.5}{.2}{\LARGE $S \equiv \sim (P \oplus Q)$}








\normalpage{cs113 Lab2: Problem 2.10}{
\qi{Create the Circuit for the given Boolean Function for Problem 2.10}



\begin{tabular}{|c|c|c||c|}
\hline
$ P $ & $ Q $ & $ R $ & $ ((P \wedge Q) \vee ((Q \wedge R) \wedge (Q \vee R))) $ \\
\hline
0 & 0 & 0 & 0 \\
0 & 0 & 1 & 0 \\
0 & 1 & 0 & 0 \\
0 & 1 & 1 & 1 \\
1 & 0 & 0 & 0 \\
1 & 0 & 1 & 0 \\
1 & 1 & 0 & 1 \\
1 & 1 & 1 & 1 \\
\hline
\end{tabular}
\label{table:tt1}





\begin{GFT}{SML}
\+fun p210(P,Q,R) = OR(AND(P,Q),AND(AND(Q,R),OR(Q,R)));\\
\+val truthValues3 = [(0,0,0),(0,0,1),(0,1,0),(0,1,1),\\
\+(1,0,0),(1,0,1),(1,1,0),(1,1,1)];\\
\+map p210 truthValues3;\\
\end{GFT}
}
\putfig{.6}{.6}{.3}{p210}
\putfig{.45}{.2}{.5}{p210SML}
\btext{.44}{.5}{.4}{\LARGE $S \equiv ((P \wedge Q) \vee ((Q \wedge R) \wedge (Q \vee R)))$}








\normalpage{cs113 Lab2: Problem 2.14}{
\qi{Prove the two Circuit are equivalent for Problem 2.14}

\begin{tabular}{|c|c||c|}
\hline
$ P $ & $ Q $ & $ ((P \wedge Q) \vee ( \sim P \wedge  \sim Q)) $ \\
\hline
0 & 0 & 1 \\
0 & 1 & 0 \\
1 & 0 & 0 \\
1 & 1 & 1 \\
\hline
\end{tabular}
\label{table:tt1}




\begin{tabular}{|c|c||c|}
\hline
$ P $ & $ Q $ & $  \sim (P \oplus Q) $ \\
\hline
0 & 0 & 1 \\
0 & 1 & 0 \\
1 & 0 & 0 \\
1 & 1 & 1 \\
\hline
\end{tabular}
\label{table:tt1}

\qi{Since the last column of both tables are identical \\
the two expressions are equivalent}

}
\putfig{.5}{.2}{.3}{p214a}
\putfig{.5}{.6}{.3}{p214b}



\normalpage{cs113 Lab2: Problem 2.14 (Cont.)}{
\qi{Prove the two Circuits are equivalent for Problem 2.14}



\begin{GFT}{SML}
\+fun p214a(P,Q) = OR(AND(P,Q),AND(NOT(P),NOT(Q)));\\
\+val truthValues2 = [(0,0),(0,1),(1,0),(1,1)];\\
\+map p214a truthValues2;\\
\+\\
\+\\
\+\\
\+\\
\+\\
\+\\
\+fun p214b(P,Q) = NOT(XOR(P,Q));\\
\+val truthValues2 = [(0,0),(0,1),(1,0),(1,1)];\\
\+map p214b truthValues2;\\
\end{GFT}
}
\putfig{.45}{.3}{.5}{p214aSML}
\putfig{.45}{.6}{.5}{p214bSML}
\btext{.1}{.5}{.47}{\LARGE $(P \wedge Q) \vee ( \sim P \wedge  \sim Q) \equiv \sim (P \oplus Q)$}




\end{comment}
\end{document}


